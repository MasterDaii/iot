\documentclass[12pt]{report}
\title{\Huge\textbf{Sprawozdanie}}
\usepackage[T1]{fontenc}
\usepackage[polish]{babel}
\usepackage[utf8]{inputenc}
\usepackage{lmodern}
\selectlanguage{polish}
\usepackage{color}

\begin{document}
\maketitle
\begin{center}
\begin{tabular}{|c|}
\hline 
\textbf{Politechnika Świętokrzyska w Kielcach}  \\
\textbf{Wydział Elektrotechniki, Automatyki i Informatyki} \\
\hline 
\textbf{Laboratorium} : Technologie loT rozproszone sieci sensor \\
\hline
\textbf{Autor}: Wojciech Harabin\\
\textbf{Grupa}: 3ID15B\\
\hline
\textbf{Laboratorium}:1\\
\hline
\end{tabular}
\end{center}
\section*{Wprowadzennie}\label{wprowadzenie}
Na pierwszych laboratorium przedstawione zostały przez prowadzącego
podstawowe zagadnienia związane z przedmiotem oraz zasady zaliczenia. Środowiskiem 
w którym wykonywane były zadania jest LateX. 
Jest to program do zautomatyzowanego składu tekstu, służący do formatowania 
dokumentów tekstowych i tekstowo-graficznych. 
\section*{Zadania}\label{zadania}
\subsection*{Tworzenie Tabeli:}
\begin{tabular}{|l|l|c|}
\hline \hline
Imie & Nazwisko & Grupa \\
\hline
Kacper & Wojtkowiak & 11B \\
\hline
Michał & Adamiak & 15A \\
\hline
Jarek & Nowak & 13B \\
\hline
Jacek & Kowalski & 13B \\
\hline \hline
\end{tabular}
\begin{tabular}{|lc|r|}
\hline
Imie  & Nazwisko & Grupa\\
\hline
Janek & Bednar &  10A\\
Wojtek & Kowalski & 11D\\
Magda & Malczyk & 15A\\
Weronika & Góral & 13C\\
\hline
\hline
Kierunek:&
Informatyka &
1 semstr\\
\hline
\end{tabular}
\newpage
\subsection*{Kolorowanie tekstu:}
\fboxrule=0.5mm
\fcolorbox{green}{red}{Technologie loT rozproszone sieci sensor}\\
\textcolor{green}{Technologie loT rozproszone sieci sensor} \\
\colorbox{yellow}{Technologie} lot  \colorbox{red}{rozproszone} sieci\colorbox{green}{sensor}.\\
Technologie \textcolor{red}{loT} rozproszone \textcolor{blue}{sieci} sensor. \\
\subsection*{Formatowanie czcionki:}
\textit{Technologie loT rozproszone sieci sensor} \\
\textbf{Technologie loT rozproszone sieci sensor} \\
\textsf{Technologie loT rozproszone sieci sensor} \\
\textsl{Technologie loT rozproszone sieci sensor} \\
\subsection*{Wnioski}
Obsługa oprogramowaniem LaTeX jest prosta i przejrzysta co więcej pozwala on na zaoszczędzenie dużej ilości czasu w porównaniu do innych oprogramowań.
Sam program oferuje wiele przydatnych funkcji, interfejs jest przyjazny dla
użytkownika dlatego korzystanie z niego jest przyjemne. 

\begin{thebibliography}{}

\bibitem{}
http://www.cs.put.poznan.pl/ksiek/pi/latex.html
\bibitem{}
http://latex-kurs.x25.pl//
\bibitem{}
https://www.latex-project.org/
 \end{thebibliography}
\end{document}
